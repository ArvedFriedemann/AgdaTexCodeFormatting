
\usepackage{listings}
\usepackage{xcolor}
\RequirePackage{fontspec}
\usepackage[framemethod=tikz]{mdframed}
\usepackage{adjustbox}
\usepackage{tikzpagenodes}
\usetikzlibrary{calc}

\newfontfamily\JuliaMonoNormal{JuliaMono-Light.ttf}[
    Path      = ../../AgdaTexCodeFormatting/juliamono/,
    Extension = .ttf
    ]

\newfontfamily\JuliaMonoItalic{JuliaMono-LightItalic.ttf}[
    Path      = ../../AgdaTexCodeFormatting/juliamono/,
    Extension = .ttf
    ]
    
\newfontfamily\JuliaMonoBold{JuliaMono-SemiBold.ttf}[
    Path      = ../../AgdaTexCodeFormatting/juliamono/,
    Extension = .ttf
    ]
    
% \newcommand{\JuliaMonoNormal}{\ttfamily}
% \newcommand{\JuliaMonoItalic}{\ttfamily\itshape}
% \newcommand{\JuliaMonoBold}{\ttfamily\bfseries}

\newcommand{\constarr}[1]{→} %{$\rightarrow$}
\newcommand{\consttimes}[1]{×} %{$\times$}
\newcommand{\constnat}[1]{$\NN$} %{ℕ} %{$\NN$}
\newcommand{\constbool}[1]{$\BB$} %{𝔹} %{$\BB$}
\newcommand{\constarrleft}[1]{←} %{$\leftarrow$}
\newcommand{\constsigma}[1]{Σ} %{$\Sigma$}
\newcommand{\consttimesSL}[1]{{\JuliaMonoNormal{\_}×}} %{\JuliaMonoNormal{\_}$\times$}
\newcommand{\consttimesSR}[1]{{×\JuliaMonoNormal{\_}}} %{$\times$\JuliaMonoNormal{\_}}
\newcommand{\consttimesSLSR}[1]{\JuliaMonoNormal{\_}×\JuliaMonoNormal{\_}} %{\JuliaMonoNormal{\_}$\times$\JuliaMonoNormal{\_}}
\newcommand{\constforall}[1]{∀} %{$\forall$}
\newcommand{\consttop}[1]{⊤} %{$\top$}
\newcommand{\constexists}[1]{∃} %{$\exists$}
\newcommand{\constimpl}[1]{⇒} %{$\Rightarrow$}
\newcommand{\constbiimpl}[1]{⇐} %{$\Leftrightarrow$}
\newcommand{\constequiv}[1]{≡} %{$\equiv$}
\newcommand{\constentails}[1]{⊨} %{$\models$}
\newcommand{\constleq}[1]{≤} %{$\leq$}
\newcommand{\constgeq}[1]{≥} %{$\geq$}
\newcommand{\constbot}[1]{⊥} %{$\bot$}
\newcommand{\constnotentails}[1] {$\not\models$} %{$\not{⊨}$}
\newcommand{\constbiimplarr}[1]{↔} %{$\leftrightarrow$}
\newcommand{\constcompose}[1]{∘} %{$\circ$}
\newcommand{\constOmega}[1]{Ω} %{$\Omega$}
\newcommand{\constRational}[1]{$\QQ$} %{ℚ} %{$\QQ$}
\newcommand{\constConj}[1]{∧} %{$\land$}
\newcommand{\constDisj}[1]{∨} %{$\lor$}
\newcommand{\constmapsto}[1]{↦} %{$\mapsto$}
\newcommand{\constepsilon}[1]{ϵ} %{$\epsilon$}
\newcommand{\constdelta}[1]{δ} %{$\delta$}
\newcommand{\consttopbot}[1]{$_\constbot^\consttop$} %{$_\bot^\top$}
\newcommand{\constinv}[1]{$^{-1}$} %{$^{-1}$}
\newcommand{\constmu}[1]{μ} %{$\mu$}
\newcommand{\constReal}[1]{$\RR$} %{ℝ} %{$\RR$}
\newcommand{\constProbab}[1]{$\PP$} %{ℙ} %{$\PP$}
\newcommand{\constnequiv}[1]{≢} %{$\not\equiv$}
\newcommand{\constint}[1]{$\ZZ$} %{ℤ} %{$\ZZ$}
\newcommand{\constpi}[1]{Π} %{$\Pi$}


\newcommand{\codeLeftMargin}{10pt}

\lstdefinestyle{mystyle}
{
    %language = C++,
    basicstyle = {\JuliaMonoNormal},
    %backgroundcolor = {\color{back-color}},
    %stringstyle = {\color{string-color}},
    keywordstyle = {\JuliaMonoBold},
    keywordstyle = [2]{\JuliaMonoItalic},
    keywordstyle = [3]{\JuliaMonoBold\colorbox{cyan}},
    keywordstyle = [4]{\JuliaMonoNormal\constarr},
    keywordstyle = [5]{\JuliaMonoNormal\consttimes},
    keywordstyle = [6]{\JuliaMonoNormal\constnat},
    keywordstyle = [7]{\JuliaMonoNormal\constbool},
    keywordstyle = [8]{\JuliaMonoNormal\constarrleft},
    keywordstyle = [9]{\JuliaMonoNormal\constsigma},
    keywordstyle = [10]{\JuliaMonoNormal\consttimesSL},
    keywordstyle = [11]{\JuliaMonoNormal\consttimesSR},
    keywordstyle = [12]{\JuliaMonoNormal\consttimesSLSR},
    keywordstyle = [13]{\JuliaMonoNormal\constforall},
    keywordstyle = [14]{\JuliaMonoNormal\consttop},
    keywordstyle = [15]{\JuliaMonoNormal\constexists},
    keywordstyle = [16]{\JuliaMonoNormal\constimpl},
    keywordstyle = [17]{\JuliaMonoNormal\constbiimpl},
    keywordstyle = [18]{\JuliaMonoNormal\constequiv},
    keywordstyle = [19]{\JuliaMonoNormal\constentails},
    keywordstyle = [20]{\JuliaMonoNormal\constleq},
    keywordstyle = [21]{\JuliaMonoNormal\constgeq},
    keywordstyle = [22]{\JuliaMonoNormal\constbot},
    keywordstyle = [23]{\JuliaMonoNormal\constnotentails},
    keywordstyle = [24]{\JuliaMonoNormal\constbiimplarr},
    keywordstyle = [25]{\JuliaMonoNormal\constcompose},
    keywordstyle = [26]{\JuliaMonoNormal\constOmega},
    keywordstyle = [27]{\JuliaMonoNormal\constRational},
    keywordstyle = [28]{\JuliaMonoNormal\constConj},
    keywordstyle = [29]{\JuliaMonoNormal\constDisj},
    keywordstyle = [30]{\JuliaMonoNormal\constmapsto},
    keywordstyle = [31]{\JuliaMonoNormal\constepsilon},
    keywordstyle = [32]{\JuliaMonoNormal\constdelta},
    keywordstyle = [33]{\JuliaMonoNormal\consttopbot},
    keywordstyle = [34]{\JuliaMonoNormal\constinv},
    keywordstyle = [35]{\JuliaMonoNormal\constmu},
    keywordstyle = [36]{\JuliaMonoNormal\constReal},
    keywordstyle = [37]{\JuliaMonoNormal\constProbab},
    keywordstyle = [38]{\JuliaMonoNormal\constnequiv},
    keywordstyle = [39]{\JuliaMonoNormal\constint},
    keywordstyle = [40]{\JuliaMonoNormal\constpi},
    alsoletter = {\{, \}, !, -, >, <, =, |, /,\\ },
    morekeywords = {data, with, where, record, let, field, variable, open, public, pattern, do, module, constructor, renaming, instance, syntax, using, hiding, private, infix, infixr, infixl},
    morekeywords = [2]{},
    morekeywords = [3]{\{!,!\},\{!!\}, \{!0!\}, \{!1!\}, \{!2!\}, \{!3!\}, \{!4!\} },
    morekeywords = [4]{->},
    morekeywords = [5]{-x-},
    morekeywords = [6]{Nat},
    morekeywords = [7]{Bool},
    morekeywords = [8]{<-},
    morekeywords = [9]{Sigma},
    morekeywords = [10]{_-x-},
    morekeywords = [11]{-x-_},
    morekeywords = [12]{_-x-_},
    morekeywords = [13]{forall},
    morekeywords = [14]{tt},
    morekeywords = [15]{exists},
    morekeywords = [16]{=>},
    morekeywords = [17]{<=>},
    morekeywords = [18]{===},
    morekeywords = [19]{|=},
    morekeywords = [20]{leq},    
    morekeywords = [21]{geq}, 
    morekeywords = [22]{BOT},
    morekeywords = [23]{|/=},
    morekeywords = [24]{<->},
    morekeywords = [25]{o},
    morekeywords = [26]{Omega},
    morekeywords = [27]{Rational, Ratio},
    morekeywords = [28]{/\\},
    morekeywords = [29]{\\/},
    morekeywords = [30]{|->},
    morekeywords = [31]{epsilon},
    morekeywords = [32]{delta},
    morekeywords = [33]{tb},
    morekeywords = [34]{inv},
    morekeywords = [35]{mu},
    morekeywords = [36]{Real},
    morekeywords = [37]{PP},
    morekeywords = [38]{=/=},
    morekeywords = [39]{Int},
    morekeywords = [40]{Pi},
    tabsize=4,
    mathescape=true, 
    breaklines = false,
    showspaces=true,
    xleftmargin=\codeLeftMargin,
    xrightmargin=\codeLeftMargin
    %framextopmargin=10pt,
    %framexbottommargin=10pt
}

\makeatletter
\def\lst@visiblespace{\lst@ttfamily{\char32} } % <-- hack
\makeatother

%modified from https://tex.stackexchange.com/questions/116595/highlighting-haskell-listings-in-large-tex-document
\lstdefinestyle{haskellStyle}{
  frame=none,
  xleftmargin=2pt,
  %stepnumber=1,
  %numbers=left,
  %numbersep=5pt,
  %numberstyle=\JuliaMonoNormal\tiny\color[gray]{0.3},
  belowcaptionskip=\bigskipamount,
  captionpos=b,
  escapeinside={*'}{'*},
  language=haskell,
  tabsize=2,
  emphstyle={\bf},
  commentstyle=\it,
  stringstyle=\JuliaMonoNormal,
  showspaces=false,
  keywordstyle=\JuliaMonoBold,
  %columns=flexible,
  basicstyle=\JuliaMonoNormal,
  showstringspaces=false,
  morecomment=[l]\%,
  keywordstyle = [4]{\JuliaMonoNormal\constarr},
  morekeywords = [4]{->},
  deletekeywords={read},
  mathescape=true,
}

\newlength{\whatsleft}
%measure the remainding length of text
\newcommand{\measureremainder}[1]{%
\begin{tikzpicture}[overlay,remember picture]
    % Measure distance to right text border
    \path let \p0 = (0,0), \p1 = (current page text area.east) in
        [/utils/exec={\pgfmathsetlength#1{\x1-\x0}\global#1=#1}];
\end{tikzpicture}%
}
%Usage:
% \measureremainder{\whatsleft} and then use \whatsleft where you need it


%These commands should be in the code-environment, but that'd mean that they couldn't be used inline
\newcommand{\arr}{\rightarrow}
\newcommand{\arrleft}{\leftarrow}
\newcommand{\impl}{\Rightarrow}
\renewcommand{\iff}{\Leftrightarrow}
\newcommand{\fmap}{<\!\!\$\!\!>}

\newcommand{\adjustboxWidth}{\linewidth-\codeLeftMargin-\codeLeftMargin-\codeLeftMargin-\codeLeftMargin}

\lstnewenvironment{code}{
	\lstset{style = mystyle}
	\minipage[t]{\linewidth}
  \measureremainder{\whatsleft}%problem: measurement can interfere with the page!
  \adjustbox{max width=\whatsleft-\codeLeftMargin, margin = \codeLeftMargin 0ex}\bgroup
	}{
  \egroup
	\endminipage
	}

  %for when the measurement until the line end breaks things
\lstnewenvironment{unmeasuredcode}{
  \lstset{style = mystyle}
  \minipage[t]{\linewidth}
  \adjustbox{max width=\linewidth-\codeLeftMargin, margin = \codeLeftMargin 0ex}\bgroup
  }{
  \egroup
  \endminipage
  }

\lstnewenvironment{pseudocode}{
  \lstset{style = mystyle}
  \minipage[t]{\linewidth}
  \vspace*{2pt}
  \hspace*{2pt}
  \textit{pseudocode:}\\
  \measureremainder{\whatsleft}%\adjustboxWidth
  \adjustbox{max width=\whatsleft-\codeLeftMargin, margin = \codeLeftMargin 0ex}\bgroup
  }{
  \egroup
  \endminipage
  }

\lstnewenvironment{codent}{
  \lstset{style = mystyle}
  \minipage[t]{\linewidth}
  \vspace*{2pt}
  \hspace*{2pt}
  \textit{no termination check:}\\
  \measureremainder{\whatsleft}%\adjustboxWidth
  \adjustbox{max width=\whatsleft-\codeLeftMargin, margin = \codeLeftMargin 0ex}\bgroup
  }{
  \egroup
  \endminipage
  }
	
\lstnewenvironment{codeerr}{
  \lstset{style = mystyle}
  \minipage[t]{\linewidth}
  \vspace*{2pt}
  \hspace*{2pt}
  \textit{error:}\\
  \measureremainder{\whatsleft}%\adjustboxWidth
  \adjustbox{max width=\whatsleft-\codeLeftMargin, margin = \codeLeftMargin 0ex}\bgroup
  }{
  \egroup
  \endminipage
  }

\surroundwithmdframed[
  %hidealllines=true,
  %backgroundcolor=lightgray,
  %outerlinewidth=2,
  leftmargin=5,
  rightmargin=5,
  outerlinecolor=lightgray,
  innerleftmargin=0pt,
  innertopmargin=0pt,
  innerbottommargin=0pt]{code}

\surroundwithmdframed[
  %hidealllines=true,
  %backgroundcolor=lightgray,
  %outerlinewidth=2,
  leftmargin=5,
  rightmargin=5,
  outerlinecolor=lightgray,
  innerleftmargin=0pt,
  innertopmargin=0pt,
  innerbottommargin=0pt]{unmeasuredcode}

\definecolor{ndetalert}{RGB}{255, 135, 98}
\definecolor{ndetalertbw}{RGB}{135, 135, 135}

\surroundwithmdframed[
  %hidealllines=true,
  %backgroundcolor=lightgray,
  %outerlinewidth=2,
  leftmargin=5,
  rightmargin=5,
  %outerlinecolor=ndetalert,
  innerleftmargin=0pt,
  innertopmargin=0pt,
  innerbottommargin=0pt]{codent}

\surroundwithmdframed[
  %hidealllines=true,
  backgroundcolor=ndetalert,
  %outerlinewidth=2,
  leftmargin=5,
  rightmargin=5,
  %outerlinecolor=ndetalert,
  innerleftmargin=0pt,
  innertopmargin=0pt,
  innerbottommargin=0pt]{codeerr}

\surroundwithmdframed[
  %hidealllines=true,
  %backgroundcolor=lightgray,
  %outerlinewidth=2,
  leftmargin=5,
  rightmargin=5,
  outerlinecolor=lightgray,
  innerleftmargin=0pt,
  innertopmargin=0pt,
  innerbottommargin=0pt]{pseudocode}
  
\lstnewenvironment{plaincode}{
	\lstset{style = mystyle}
	\minipage[t]{\linewidth}
  \measureremainder{\whatsleft}%\adjustboxWidth
  \adjustbox{max width=\whatsleft, margin = 0pt 0pt}\bgroup
	}{
  \egroup
	\endminipage
	}

	
\newsavebox{\mybox}		
\lstnewenvironment{AgdaOutput}{
	\lstset{style = mystyle} %  backgroundcolor=\color{lightgray}
	\minipage{\linewidth}
  \adjustbox{max width=\adjustboxWidth, margin = \codeLeftMargin 0ex}\bgroup
	}{
  \egroup
	\endminipage
	}


\surroundwithmdframed[
  %hidealllines=true,
  backgroundcolor=lightgray,
  %outerlinewidth=2,
  leftmargin=5,
  rightmargin=5,
  outerlinecolor=lightgray,
  innerleftmargin=0pt,
  innertopmargin=0pt,
  innerbottommargin=0pt]{AgdaOutput}

\lstnewenvironment{haskell}{
	\lstset{style = haskellStyle}
	\minipage{\linewidth}
  \adjustbox{max width=\adjustboxWidth, margin = \codeLeftMargin 0ex}\bgroup
	}{\egroup\endminipage}
	
\surroundwithmdframed[
  %hidealllines=true,
  %backgroundcolor=lightgray,
  %outerlinewidth=2,
  leftmargin=5,
  rightmargin=5,
  outerlinecolor=lightgray,
  innerleftmargin=0pt,
  innertopmargin=0pt,
  innerbottommargin=0pt]{java}
	
\lstnewenvironment{java}{
	\lstset{style = haskellStyle, language = Java}
	\minipage{\linewidth}
  \adjustbox{max width=\adjustboxWidth, margin = \codeLeftMargin 0ex}\bgroup
	}{\egroup\endminipage}
	
%\lstMakeShortInline[columns=fixed, style = mystyle]|

\newcommand{\inlcode}[1]{\lstinline[columns=fixed, style = mystyle]|#1|}
\newcommand{\ic}[1]{\inlcode{#1}}
\newcommand{\hscode}[1]{\lstinline[columns=fixed, style = haskellStyle]|#1|}
